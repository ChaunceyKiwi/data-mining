\documentclass[../notes.tex]{subfiles}

\begin{document}

\section{Recommender Systems}

\subsection{Introduction}
\subsubsection{Motivation}
\begin{itemize}
  \item Users want to have personalized results, but are not willing to spend a lot of time to specify their personal information needs.
  \item Recommender systems automatically identify information relevant for a given user, learning from available data.
  \item Data
  \begin{itemize}
    \item user behaviour: Boolean data (clicks, views, purchases), Integer data (ratings)
    \item user profiles: (demographic attributes, list of interests, ...)
  \end{itemize}
\end{itemize}

\subsubsection{Tasks}
\begin{itemize}
  \item Rating prediction: Predict the rating of target user for target item, e.g. predict Joe’s rating for Titanic.
  \item Top-N item recommendation: Predict the top-N highest-rated items among the items not yet rated by target user.
  \item Friend recommendation (only for social network): Predict the top-N users to which the target user is most likely to connect.
\end{itemize}

\subsubsection{Performance Evaluation}
\begin{itemize}
  \item Cross-validation on offline dataset
  \begin{itemize}
    \item Withold subset of ratings (test set)
    \item Use remaining ratings to train recommender (traning set)
    \item Compare the withold ratings against the predicted ratings, compute measure of prediction error
  \end{itemize}

  \item A/B test with online system
  \begin{itemize}
    \item Assign users randomly to one of two algorithms
    \item Compute measure of business value, e.g. click-through rate, conversion rate, return rate of customers and profit
  \end{itemize}

  \item Measures for rating prediction
  \begin{itemize}
  \item Mean absolute error
    $$MAE = \frac{1}{|Test|} \sum_{(u, i) \in Test} |\hat r_{u,i} - r_{u, i}|$$
    \item Root mean square error
    $$RMSE = \sqrt{\frac{\sum_{(u, i) \in Test} (\hat r_{u,i} - r_{u, i})^2}{|Test|}}$$
  \end{itemize}

  \item Measures for top-N recommendation
  \begin{itemize}
    \item TopN: set of the top-N recommendations (by algorithm)
    \item TestTop: set of all elements of the test set that are among the top-N items for the user

    $$Recall = \frac{|TopN \cap TestTop|}{|TestTop|}$$
  \end{itemize}
\end{itemize}

\subsection{Content-based Recommendation}
\subsubsection{Introduction}
\begin{itemize}
  \item Set of items I, set of users U
  \item User profiles: describing the users' tastes, preferences and needs 
  \item Item profiles: characterizing the content of item
  \item Approach1: Boolean rating prediction can be formulated as a classification task, but unusual approach.
  \item Approach2: Top-N recommendation by ranking items w.r.t. similarity of item profiles and user profile 
\end{itemize}

\subsubsection{Top-N Recommendation}
\textbf{Item Profile}
\begin{itemize}
  \item Item profile: typically frequencies of k selected keywords.
  \item $f_{i,j}$: frequency of keyword $i$ in item $j$
  \item $n_i$: number of items contaning keyword $i$
  \item Term frequency $$TF_{i,j} = \frac{f_{i,j}}{max_z f_{z,j}}$$
  \item Inverse document frequcny $$IDF_i = log(\frac{M}{n_i})$$
  \item $$w_{i,j} = TF_{i,j} \cdot IDF_i$$
  \item Profile for item $i$: $$content(i) = (w_{1,i}, ..., w_{k,i})$$
\end{itemize}

\textbf{User Profile}
\begin{itemize}
  \item Typically importance or frequencies of keywords, e.g. aggregation of profiles of items liked by user. $$contentBasedProfile(u) = (w_{1,u}, ..., w_{k,u})$$
  \item Similarity of user $i$ and user $u$: $$sim(u,i) = \frac{\sum_{l=1}^{k} w_{l,u} w_{l,i}}{\sqrt{\sum_{l=1}^{k} w_{l,u}^2}\sqrt{\sum_{l=1}^{k} w_{l,i}^2}}$$
\end{itemize}

\subsection{Collaborative Filtering}
\subsubsection{Introduction}
\begin{itemize}
  \item Set of items I, set of users U
  \item User rate items
  \item No need for information about content of items or attributes of users (good for privacy).
  \item Users with similar ratings on some items are likely to have similar ratings on further item.
  \item Items which are rated similarly by some users are likely to have similar ratings by further users.
  \item Two paradigms: memory-based (lazy-learning) and model-based (eager learning)
\end{itemize}

\subsubsection{Memory-based Methods}
\begin{itemize}
  \item User-based CF
  \begin{itemize}
    \item Find users with similar rating profiles
    \item Aggregate their ratings for item i to predict unknown rating $r_{u,i}$
  \end{itemize}

  \item Item-based CF
  \begin{itemize}
    \item Find items with similar rating profiles
    \item Aggregate their ratings by user u to predict unknown rating  $r_{u,i}$
  \end{itemize}

  \item Issues
  \begin{itemize}
    \item How to define user/item similarity?
    \item How many similar users/items?
    \item How to aggregate the ratings?
  \end{itemize}

  \item Definition
  \begin{itemize}
    \item $r_{u,i}:$ (observed) rating of user $u$ for item $i$
    \item $\bar r_u:$ mean rating of user $u$
    \item $\hat r_{u,i}:$ predicted rating of user $u$ for item $i$
    \item $N(u):$ set of users similar to user $u$ (who have rated item $i$)
    \item $sim(u,v):$ similarity of users $u$ and $v$
    \item $\kappa:$ nomarlization factor
  \end{itemize}

  \item Different users use the ratings scale differently, thus we need to normalize ratings by the mean rating of a user/item
  \item The more similar a user/item, the higher the weiht of the rating
  \item Rating prediction for user-based CF $$\hat r_{u,i} = \bar r_{u} + \kappa \sum_{v \in N(u)}sim(u,v)(r_{v,i} - \bar r_v)$$

  \item How to define similarity of users
  \begin{itemize}
    \item $I_{uv}:$ set of items rated by both users $u$ and $v$
    \item Pearson correlation coefficient
    $$sim(u,v) = \frac{\sum_{i \in I_{uv}} (r_{u,i} - \hat r_u) ((r_{v,i} - \hat r_v))}{\sqrt{\sum_{i \in I_{uv}} (r_{u,i} - \hat r_u)^2 ((r_{v,i} - \hat r_v))^2}}$$
    \item Cosine similarity
    $$sim(u,v) = \frac{\sum_{i \in I_{uv}} r_{u,i} r_{v,i}}{\sqrt{\sum_{i \in I_{uv}} r_{u,i}^2 \sqrt{\sum_{i \in I_{uv}} r_{v,i}^2}}}$$
  \end{itemize}

  \item Efficiency issues
  \begin{itemize}
    \item Computation of the $k$ most similar users/items is expensive.
    \item Without index support, runtime is $O(n)$
  \end{itemize}

  \item Clustering-based approach
  \begin{itemize}
    \item Cluster items/users
    \item Data structure for efficient lookup of clusters
    \item Aggregate the ratings within the user/item cluster
  \end{itemize}

  \item Modification of k-means
  \begin{itemize}
    \item When computing centroids, consider only the known ratings.
    \item When computing distances to centroids, consider only known ratings and normalize by the number of those ratings.
  \end{itemize}
\end{itemize}

\subsection{Matrix Factorization}

\textbf{Introduction}
\begin{itemize}
  \item A model-based approach to CF, which assumes that the latent factors represent unobserved preferences of users and characteristics of items.
  \item Non-negative MF \\ $R \approx UV^T$, where U are the user factors and V the item factors
  \item Objective $$ argmin_{U,V}||R - UV^T||^2, U \ge 0, V \ge 0$$ where $||.||^2$ denotes the squared Frobenius norm.
\end{itemize}

\subsubsection{Probabilistic MF}
\begin{itemize}
  \item Accurate and efficient rating prediction for sparse datasets
  \item Assume that ratings are generated from a linear probabilistic model with Gaussian observation noise:
  $$p(R|U,V,\sigma^2) = \prod_{i=1}^{N} \prod_{i=1}^{M}$$
\end{itemize}

\subsection{Link Recommendation}
\subsubsection{Introduction}

\textbf{Problem definitions}
\begin{itemize}
  \item Given a user pair (u,v), estimate the probability of creation of the link $u \rightarrow v$.
  \item Given a user $u$, recommend a list of top users for $u$ to connect to
\end{itemize}

\subsubsection{Memory-based Methods}
\begin{itemize}
  \item Explore the social network starting from user $u$, for which links shall be predicted
  \item Rank links to users $v$ based on the network-based similarity between $u$ and $v$.
\end{itemize}

\textbf{Topology-based Methods}
\begin{itemize}
  \item Measure similarity based on the direct neighbors of users $u$ and $v$ (local measure).
  \item Common neighbors: $$score(A,B) = |N_A \cap N_B|$$
  \item Jaccard coefficient: $$score(A, B) = \frac{|N_A \cap N_B|}{|N_A \cup N_B|}$$
  \item Adamic \& Adar score: $$score(A,B) = \sum_{C \in N_A \cap N_B} \frac{1}{log|N_C|}$$
  \item Preferential attachment (initially propsed for modeling network growth) $$score(A,B) = |N_A|\cdot|N_B|$$
  \item SimRank (two users are similar to the extent that they are joined to similar neighbors)
  $$score(A,B) = \gamma \frac{\sum_{x \in N_A} \sum_{y \in N_B} score(x, y)}{|N_A| \cdot |N_B|}$$
  $$score(A,B) = 1$$
\end{itemize}

\textbf{Path-based Methods}
\begin{itemize}
  \item Measure similarity, based on number of paths between $A$ and $B$
  \item $score(A,B)$: average number of steps for a random walk from A to B $$score(A,B) = \sum_{l=1}^{\infty} (\beta^l\cdot paths^{l}_{A,B})$$
  \item where $paths^{l}_{A,B}$ is the number of paths of length $l$ from $A$ to $B$ 
  \item Typically consider only paths up to a certain maximum length (global measure)
  \item More accurate but less efficient
  \item Random walk with restart
  \begin{itemize}
    \item A random walk starts from user $A$
    \item At each step, with probability $\alpha$ the random walk restarts
    \item $score(A,B)$: probability of being at user $B$ during random walk from $A$
  \end{itemize}
\end{itemize}

\subsubsection{Model-based Methods}
\begin{itemize}
  \item Learn a model that explains the generation of new links
  \item Typically consider only paths to a certain maximum length (global measure)
\end{itemize}

\textbf{Matrix Factorization}
\begin{itemize}
  \item Model-based approach to $CF$
\end{itemize}


\subsection{Research Issues in Recommender Systems}

\textbf{Recommendation in Social Networks}
\begin{itemize}
  \item Social influence and selection
  \begin{itemize}
    \item Preferences of friends are more likely to be similar than those of two random users
    \item How to model social influence and selection in a recommender system
  \end{itemize}

  \item Social influence can be context-specific
  \item Distinguish strong and weak ties:
  \begin{itemize}
    \item Strong ties: within community
    \item Weak ties: across communities
  \end{itemize}
\end{itemize}

\textbf{Recommendation in Location-based Networks}
\begin{itemize}
  \item Users and items are embedded in a geographical (2-dimensional) space.
  \item Geographical influence
  \begin{itemize}
    \item Users are more likely to check-in at nearby locations
    \item Nearby locations are similar to each other
  \end{itemize}

  \item Friends may not live near to each other
  \begin{itemize}
    \item They have similar interests.
    \item But they may check-in at different locations
  \end{itemize}

  \item How to model geographical influence in a recommender system?
\end{itemize}

\textbf{Privacy-Preserving Recommendation}
\begin{itemize}
  \item Recommender systems allow effective personalization of information.
  \item But recommender systems threaten the privacy of users, since they often use private user data.
  \item User should be informed about the privacy implications and should be able to choose their own trade-off between personalization and privacy
  \item How can recommender systems inform users in an understandable manner about privacy issues and suppport different tradeoffs for different users?
\end{itemize}

\end{document}