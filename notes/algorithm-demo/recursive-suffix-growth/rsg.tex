\documentclass[12pt]{article}
\usepackage[utf8]{inputenc}
\usepackage[a4paper, top=1in, bottom=1in, left=1in, right=1in]{geometry} 
\usepackage{graphicx}
\setlength\parindent{0pt} % Removes all indentation from paragraphs - comment this line for an assignment with lots of text

\usepackage{algorithm}
\usepackage[noend]{algpseudocode}

\usepackage{amsmath}
\usepackage{amssymb}
\usepackage{tikz}
\usepackage{pgfplots}

\newcommand\tab[1][1cm]{\hspace*{#1}}
\newcommand{\vect}[1]{\boldsymbol{#1}}

\title{\huge{\textbf{Demo of Algorithm Recursive Suffix Growth}}}
\author{Chauncey Liu}
\date{\today}

\begin{document}
 
\maketitle

\section{Recursive Suffix-based Pattern Growth}
\begin{itemize}
\item In order not to waste the computational effort of counting, form projected database for a frequent itemset P: all transactions containing itemset $P$
\item If a transaction does not contain the itemset corresponding to an enumeration-tree node, then this will not be relevant for counting at any descendent (superset itemset) of tht node.
\item Count support of extensions of P only in projected database of P.
\item Use absolute minsup, not relative minsup.
\item Start with empty pattern (suffix) and complete database $D$, where D has been filtered to contain only frequent items.
\item Recursive calls for all extensions and their projected databases.
\item The items in the database are ordered with decreasing support. The lexicographic ordering is used to define the ordering of items within itemsets and transactions. 
\end{itemize}

\begin{algorithm}
\caption{Algorithm Recursive Suffix Growth}
\begin{algorithmic}[0]
\State /* D: transactions in terms of frequent 1-items, i.e. without infrequent items */
\State /* P: current suffix itemset */
\State /* reports all frequent itemsets with suffix $P$ */ \\

\Function{RecursiveSuffixGrowth}{D, minsup, P}
\For{\textbf{each} item \textbf{i} in any transaction of D}
  \State \textbf{report} itemset $P_i = \{i\} \cup P$  as frequent;
  \State Form $D_i$ with all transactions from $D$ containing item $i$;
  \State Remove all items from $D_i$ that are lexicographically $y \ge i$;
  \State Rmove all infrequent items from $D_i$
  \If{$D_i \ne \varnothing$}
    RecursiveSuffixGrowth($D_i, minsup, P_i$)
  \EndIf
\EndFor
\EndFunction
\end{algorithmic}
\end{algorithm}

\newpage
 
\section{Demo}

\begin{tabular}{| c | c |}
\hline
ID & Transaction \\
\hline
1 & \{f, a, c, d, g, i, m, p\} \\
2 & \{a, b, c, f, l, m, o\} \\
3 & \{b, f, h, j, o, w\} \\
4 & \{b, c, k, s, p\} \\
5 & \{a, f, c, e, l, p, m, n\} \\
\hline
\end{tabular} \\

Scan DB once, find single item frequent pattern with $minsup$ = 3: \\
f: 4 \\
c: 4 \\
a, b, m, p: 3 \\

Transactions in terms of frequent 1-items, i.e. without infrequent items: \\
\{a, c, f, m, p\} \\
\{a, b, c, f, m\} \\
\{b, f\} \\
\{b, c, p\} \\
\{a, c, f, m, p\} \\

In the beginning, $P = \varnothing$, item $\textbf{i} = f$ \\

1. Report $P_i = \{f\} \cup \varnothing = \{f\}$ as frequent \\

2. Form $D_i$ with all transactions from $D$ containing item $f$: \\
$D_i = $ \\ 
\{a, c, f, m, p\} \\
\{a, b, c, f, m\} \\
\{b, f\} \\
\{a, c, f, m, p\} \\

3. Remove all items from $D_i$ that are lexicographically $\ge i$ \\
$D_i = \{\{a, c\}, \{a, b, c\}, \{b\}, \{a, c\}\}$ \\

4. Remove all infrequent items from $D_i$ \\
$D_i = \{\{a, c\}, \{a, c\},\{a, c\}\}$ \\

5. Run RecursiveSuffixGrowth($\{\{a, c\}, \{a, c\},\{a, c\}\}$, 3, $\{f\}$), go to 6 and 7. \\

6. Report $\{a, f\}$, $D_i = \varnothing$, return. \\

7. Report $\{c, f\}$, $D_i = \{\{a\},\{a\},\{a\}\}$, run RecursiveSuffixGrowth($\{\{a\}, \{a\},\{a\}\}$, 3, $\{cf\}$), goto 8. \\

8. Report $\{a, c, f\}$, $D_i = \varnothing$, return. \\

\textbf{All frequent itemsets with suffix as $f$ are found (f, af, cf, acf). Continue with other items.}
\end{document}